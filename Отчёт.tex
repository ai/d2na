\documentclass[a4paper,14pt]{extarticle}
\usepackage[utf8]{inputenc}
\usepackage[russian]{babel}
\renewcommand{\baselinestretch}{1.5}

\begin{document}

\section*{Аннотация}

Цель работы — создание инструментов для автоматического написания программ с
помощью генетических алгоритмов.

Современные программы не изменяют сами себя. Практически весь процесс создания
ПО привязан к человеку-разработчику. Однако наблюдение за живой природой
показывают нам, что объект может динамически саморазвиваться, чтобы лучше
соответствовать внешним условиям и поставленным перед ним целям.

Органическая жизнь саморазвивается с помощью эволюции, биологической реализации
генетических алгоритмов. Они активно применяются в информационных технологиях и
изучены на достаточном уровне, чтобы применить их для саморазвития алгоритмов
программного обеспечения.

В основе генетических алгоритмов лежат случайные изменения, смешивания и отбор.
К сожалению, реализация первых двух действий для исходного или машинного кода 
программ неэффективна. Поэтому выгодно представить алгоритм в более удобном
виде.

В данной работе будут рассмотрены существующие способы эволюции алгоритмов и
показан новый способ, более эффективный для представленных целей.

\newpage
\tableofcontents

\end{document}
