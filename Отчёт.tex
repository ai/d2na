\documentclass[a4paper,14pt]{extarticle}
\usepackage[utf8]{inputenc}
\usepackage[russian]{babel}

\renewcommand{\baselinestretch}{1.5}

\usepackage{geometry}
\geometry{left=3cm}
\geometry{right=2cm}
\geometry{top=2cm}
\geometry{bottom=2cm}

\begin{document}

\section*{Аннотация}

Цель работы — создание инструментов для автоматического написания программ с
помощью генетических алгоритмов.

Современные программы не изменяют сами себя. Практически весь процесс создания
ПО привязан к человеку-разработчику. Однако наблюдение за живой природой
показывают нам, что объект может динамически саморазвиваться, чтобы лучше
соответствовать внешним условиям и поставленным перед ним целям.

Органическая жизнь саморазвивается с помощью эволюции, биологической реализации
генетических алгоритмов. Они активно применяются в информационных технологиях и
изучены на достаточном уровне, чтобы применить их для саморазвития алгоритмов
программного обеспечения.

В основе генетических алгоритмов лежат случайные изменения, смешивания и отбор.
К сожалению, реализация первых двух действий для исходного или машинного кода 
программ неэффективна. Поэтому выгодно представить алгоритм в более удобном
виде.

В данной работе будут рассмотрены существующие способы эволюции алгоритмов и
показан новый способ, более эффективный для представленных целей.

\newpage
\tableofcontents

\newpage
\section{Постановка задачи}
Разработчик формулирует требование к алгоритму $a$ в виде оценочной
функции $T$, так чтобы более подходящий для задачи алгоритм давал больший
результат $T(a)$. Разработчик также предоставляет функцию останова $F$, такую,
что $F(T(a))$ вернёт 1, если алгоритм $a$ полностью подходит и цель разработки
достигнута.

Цель данной работы, предоставить функцию $G(T, F) = a$, которая вернёт алгоритм подходящий разработчику.

Важным требованием в работе является то, чтобы итоговый алгоритм был понятен
человеку и весь процесс его генерации можно было наглядно представить и
анализировать.

\newpage
\section{Генетические алгоритмы}
Как мы видим, задача сводится к оптимизации функции $T(a)$. Однако характер этой
функции неизвестен. Для подобных задач оптимизации весьма
эффективны генетические алгоритмы.

Генетический алгоритм $G_{ga}$ работает сразу с множеством кандидатов на решение
(популяцией) $A = [a_1, a_2, …, a_n]$. Это итерационный процесс и каждый шаг
$A_i = G_{ga}(A_{i-1})$ немного улучшает оценочную функцию (за значение
оценочной функции популяции, принимается оценочная функция лучшего члена
популяции).

Каждая итерация $G_{ga}(A) = S(Mi(Mu(R(A))))$ состоит из операций:
\begin{enumerate}
  \item Размножения $R(A)$ — увеличения популяции простым копированием.
        Поскольку последующий отбор уменьшит популяцию, размножение поддерживает
        её размер на каждом шаге.
  \item Мутации $Mu(A)$ — случайного изменения элементов.
  \item Смешивания $Mi(A)$ — объединения двух элементов со случайным выбором
        частей из каждой половины.
  \item Отбора $S(A)$ — удаление наименее подходящих кандидатов. Для этого для
        каждого элемента популяции вычисляется оценочная функция и элементы с
        наименьшим её значением изымаются из множества.
\end{enumerate}

В конце каждой итерации с помощью функции отбора $F(A)$, которую задаёт разработчик,
определяется нужно ли остановить цикл.

\end{document}
